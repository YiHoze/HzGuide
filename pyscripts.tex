
\CodeSetup{language=powershell}
\RenewTerms{macros}{
    labeltype=macro,
    font=\ttfamily,
    marker={{},{}},
    delimiter=\hfill,
    labelwidth=1.5em,
    leftmargin=2.5em,
	index=false
}

\chapter{파이선 스크립트}

\begin{flushright}
\url{https://github.com/YiHoze/texwrapper}
\end{flushright}

\begin{code}
C:\>python c:\home\bin\foo.py optional_argument mandatory_argument
C:\>foo opt_arg man_arg
\end{code}

위 줄이 아니라 아래 줄과 같은 방법으로 윈도즈에서 파이선 스크립트를 실행할 수 있게 하려면 다음과 같이 설정해야 한다.

\begin{itemize}

\item \term{PATHEXT} 환경 변수에 .py를 등록한다. .py 파일들이 실행 가능한 것들로 인식된다.

\item \term{CPython}에는 \term{py.exe}\annotate{python launcher}가 포함되어 있고, py.exe가 설치되면 별다른 설정 없이도 .py 파일들이 py.exe와 연결된다. 
\term{아나콘다}\annotate{Anaconda}에는 py.exe가 포함되어 있지 않아서 별도로 설치해야 한다.
권장되는 방법이 아니지만, python.exe와 연결하려면 다음과 같이 한다.

\begin{code}
C:\>cmd /c assoc .py
C:\>cmd /c ftype Python.File="C:\Users\...\python.exe" %1 %*
\end{code}

\item .py 파일들을 \term{PATH} 환경 변수에 포함된 경로에 둔다.

\end{itemize}

레이텍 사용을 비롯하여 문서 제작에 수반되는 여러 일들을 수월하게 하고자 다음과 같은 여러 래퍼\annotate{wrapper} 스크립트가 파이선으로 작성되었다.

\begin{macros}[labelwidth=4.75em, leftmargin=5.35em, index=true]
\item[ltx.py] lualatex 또는 xelatex을 이용하여 텍 파일을 컴파일한다.
\item[i.py] 현재 폴더에서 텍 파일을 찾아 ltx.py를 이용하여 컴파일한다.
\item[iu.py] 다양한 포맷의 이미지 파일들을 다른 포맷으로 변환한다.
\item[fontinfo.py] 설치된 폰트들 또는 특정 폰트의 정보를 보여준다.
\item[tlconf.py] 텍 라이브를 위한 환경을 설정한다.
\item[op.py] 텍스트 파일들과 PDF 파일들을 미리 지정된 프로그램으로 연다.
\item[wordig.py] 특정 문자열을 찾아 다른 문자열로 바꾼다.
\item[unit.py] 특정 단위의 값을 다른 단위로 변환한다.
\item[fu.py] 파일들을 백업한다.
\end{macros}

이들 스크립트의 대부분이 공통으로 참조하는 설정 파일인 \term{docenv.conf}는 다음과 같은 섹션들을 포함하고 있다.

\begin{codewrite}
[LaTeX]
compiler = xelatex.exe

[TeX Live]
texmflocal = C:\texlive\texmf-local\tex\latex\local

[SumatraPDF]
path = C:\Program Files\SumatraPDF\SumatraPDF.exe
\end{codewrite}
\coderead[language=ini]

\section{ltx.py: 텍 파일 컴파일하기}

\term{ltx.py}는 \term{latexmk.exe}와 비슷한 레이텍 래퍼\annotate{wrapper}이다.
텍스트 에디터에서 단축키를 이용하여 텍 파일을 컴파일할 수 있지만, \macro[shell]{-shell-escape} 옵션을 주거나 xindex 또는 bibtex을 돌려야 하는 것과 같은 다양한 경우들의 각각에 매우 제한적으로 서로 다른 단축키를 할당할 수 있을 뿐이다.
명령행 사용이 거북하지 않더라도, 이렇게 명령하는 것은 다소 번거롭다.

\begin{code}
C:\>latexmk -lualatex -shell-escape foo.tex
\end{code}

latexmk를 사용하지 않은 가장 큰 이유는 초보자 시절에 그것의 사용법이 거추장스러워 보였고 그래서 자연스럽게 그것에서 멀어졌다.

\begin{code}
C:\>ltx -L -s foo.tex
\end{code}

\begin{macros}
\item[-L] docenv.conf에 설정된 디폴트 컴파일러가 무시되고, \term{lualatex}이 실행된다.
	docenv.conf가 존재하지 않으면, xelatex이 디폴트 컴파일러로 설정된다.
\item[-X] \term{xelatex}이 실행된다.
% \item[-A] docenv.conf의 \macro{alternative_path} 옵션에 설정된 폴더에 있는 컴파일러가 사용된다.
예를 들어, \macro{PATH} 시스템 변수에 \verb{C:\texlive\2022\bin\win64}를 설정하고, \macro{alternative_path} 옵션에 \verb{C:\texlive\2022\bin\win32}를 설정하라.
\item[-b] 이것은 \macro[interaction]{-interaction=batchmode}를 의미한다. 이 옵션이 주어지지 않으면 \macro[synctex]{-synctex=-1}이 사용된다.
\item[-s] 이것은 \macro[shell]{-shell-escape}와 같다.
\item[-w] 주어진 텍 파일이 두 번 컴파일된다.
\item[-f] 세 번 컴파일된다. .idx 파일이 있으면 처리된다.
\item[-v] 생성된 PDF 파일이 op.py에 의해 열린다. 
\item[-n] 컴파일이 생략된다. xindex나 bibtex 같은 다른 프로그램만 실행하려 할 때 이 옵션이 유용하다.
\item[-i] 색인이 정렬된다. 
\item[-I] 색인 정렬을 위한 언어를, \verb{german} 또는 \verb{ger}와 같이, 지정하라. 디폴트는 \verb{korean}이다. .ist 또는 .xdy를 지정하는 것도 가능하다.
\item[-a] \verb{-f} 옵션이 주어지면 컴파일 뒤에 .aux를 비롯한 모든 부수 파일들이 삭제된다. 그 파일들을 남겨두려면 이 옵션을 사용하라.
\item[-m] 모든 색인 항목들이 PDF 북마크에 추가된다.
\item[-M] 이것은 파이선 \term{독스트링}\annotate{docstring}을 위한 것이다. 색인 항목들 가운데 파이선 함수들이 PDF 북마크에 추가된다.
\item[-c] 모든 부수 파일들이 삭제된다.
\item[-B] \term{bibtex}이 실행된다.
\item[-P] \term{pythontex}이 실행된다.
\end{macros}

색인 정렬 프로그램을 선택하는 과정은 다소 복잡하다.

\begin{code}
C:\>ltx -i -I eng foo
C:\>ltx -i -I foo.xdy foo
C:\>ltx -i -I kotex.ist foo
\end{code}

지정된 언어가 미리 정의된 xindex 언어 목록에 포함되어 있으면 \term{xindex}가 사용된다.
없다면 그 언어가 xindy 언어 목록에 있는지 확인하고, 있으면 \term{texindy}가, 없으면 \term{xindex}가 영어를 기준으로 \term[idx]{.idx}를 처리한다.
\term[xdy]{.xdy} 파일이 지정되면 \term{texindy}가 사용되고, \term[ist]{.ist}가 지정되면 \term{komkindex}가 사용된다.

\begin{note}
\term{xindex}를 사용하여 한글 색인을 정렬하는 데에 \term{xindex-hz-ko.lua}가 사용되고, 그것은 \term{xindex-ko.lua}를 부른다.
\end{note}

\begin{note}
레이텍 명령들 때문에 이 문서의 색인을 정렬하는 데에 \term{hzguide.xdy}가 사용되었다.
\end{note}

\section{i.py: 더 게으른 사용자를 위한 레이텍 래퍼}

\term{i.py}는 ltx.py를 사용하는 레이텍 래퍼이다. 
인수 없이 실행될 때 이 스크립트는 다음과 같이 작동한다.

\begin{code}
C:\>i
\end{code}

\begin{enumerate}
\item 현재 폴더에서 \term{i.ini} 파일을 찾는다. 
없다면 .tex 파일들을 찾는다.
발견된 파일이 하나이면 그 파일의 이름을 아래와 같이 i.ini 파일에 기록하고 컴파일한다.

\begin{codewrite}
[tex]
target = foo.tex
\end{codewrite}
\coderead[language=ini]

\item 둘 이상의 텍 파일들이 있다면, 번호와 함께 파일 이름들이 나열된다.
번호를 입력하여 컴파일할 파일을 선택하라.
"\verb{2 5 8-10}"과 같이 복수의 파일들을 선택할 수 있다.
선택된 파일들의 이름이 i.ini에 기록되고 컴파일된다.
0을 입력하면 모든 파일들이 i.ini에 기록되고 컴파일된다.

\item i.ini 파일에 기록된 텍 파일이 하나이면 그것이 컴파일된다.
둘 이상이면, 사용자가 선택할 수 있도록 번호와 함께 파일 이름들이 나열된다. 
"\verb{2 5 8-10}"과 같이 복수의 파일들을 선택할 수 있다.
0을 입력하면 모든 파일들이 컴파일된다.
\end{enumerate}

인수가 주어졌을 때 이 스크립트는 다음과 같이 작동한다.

\begin{code}
C:\>i foo
\end{code}

\begin{enumerate}
\item 현재 폴더에 i.ini 파일이 없다면, 파일 이름에 \verb{foo}를 포함하는 텍 파일들을 찾는다.
나머지 단계는 위 과정의 첫째 및 둘째 단계와 동일한다.

\item i.ini 파일이 있고, 기록된 파일 이름들 가운데 \verb{foo}를 포함하는 것이 하나이면 그것이 컴파일된다.
나머지 단계는 위 과정의 셋째 단계와 동일하다.
\end{enumerate}

i.ini는, 다양한 방법으로 처리할 수 있도록, 여러 옵션들을 갖고 있다.

\begin{codewrite}
[tex]
target = foo.tex
compiler = -X -s
draft = wordig.py -a "..." -s "..." -v %(target)s
final = wordig.py -a "..." -s "..." -v %(target)s
final_compiler = -L -f
after = 
main = \input{preamble}
    \begin{document}
    \maketitle
    \input{\1}
    \end{document}
\end{codewrite}
\coderead[language=ini]

i.py를 다음과 같은 선택 옵션들과 함께 사용할 수 있다. i.py가 인식하지 못하는 옵션들은 ltx.py에 전달된다.
i.ini의 \macro{compiler}에 설정된 것들도 ltx.py에 전달된다.

\begin{macros}
\item[-U] i.ini를 업데이트할 수 있도록 텍 파일들을 나열한다.

\item[-D] 초본을 만들기 위한 전처리로서 i.ini의 \macro{draft}에 지정된 명령이 수행된다.
i.ini의 \macro{after}에 지정된 명령이 레이텍 컴파일이 완료된 후에 수행된다.

\item[-F] 최종본을 만들기 위한 전처리로서 i.ini의 \macro{final}에 지정된 명령이 수행된다.
그리고 \macro{final_compiler}에 설정된 옵션들이 ltx.py에 전달된다.

\item[-W] 이것은 여러 하위 파일들을 따로 컴파일하기 위한 목적으로 고안되었다.
i.ini의 \macro{main} 옵션이 위 예와 같이 설정되어야 한다. 
\verb{\1}이 선택된 파일로 바뀐다.

\begin{code}
C:\>i -W intro
\end{code}

이 예의 경우에 파일 이름에 "intro"가 포함된 텍 파일만이 삽입되고 같은 이름으로 PDF 파일이 생성된다.

\item[-N] 컴파일이 생략된다.

\item[-C] i.ini 파일이 생성된다.
\end{macros}

\section{iu.py: 이미지를 다른 포맷으로 변환하기}

\term{iu.py}에 정의된 클래스 이름이 ImageUtility인데, 그것은 억지로 지어 붙인 것이고, 이 파일 이름은 그림처럼 이쁜 아이유를 기리기 위한 것이다.
이것 역시 래퍼이기 때문에 다음과 같은 프로그램들이 필요하다.

\begin{itemize}*
\item \term{이미지매직}\annotate{ImageMagick}
\item \term{고스트스크립트}\annotate{GhostScript}
\item \term{텍 라이브}\annotate*{epstopdf.exe, pdfcrop.exe, pdftops.exe}
\item \term{잉크스케이프}\annotate{Inkscape}
\end{itemize}

iu.py는 다음과 같은 것들을 할 수 있다.

\begin{itemize}

\item 비트맵 이미지들을 다른 포맷의 이미지로 변환할 수 있다.

\begin{code}
C:\>iu -t png *.jpg
\end{code}

\item 비트맵 이미지들을 PDF 또는 EPS로 변환할 수 있다.
정확히 말해, 이미지매직이 고스트스크립트에게 그 작업을 맡긴다.

\begin{code}
C:\>iu -t pdf *.png
\end{code}

\item 벡터 이미지들을 비트맵 또는 다른 포맷의 벡터 이미지로 변환할 수 있다.

\begin{code}
C:\>iu -t png *.eps
C:\>iu -t eps *.svg
\end{code}

\item 비트맵 이미지들의 픽셀 크기나 해상도를 변경할 수 있다.

\begin{code}
C:\>iu -r -s 75 *.jpg
C:\>iu -r -d 150 *.jpg
\end{code}

\item 비트맵 이미지들의, 픽셀 크기를 포함하는, 이미지 정보를 볼 수 있다.

\begin{code}
C:\>iu -i foo.jpg
\end{code}
\end{itemize}

\begin{macros}
\item[-t] 이 옵션과 함께 지정된 포맷\annotate*{ai, eps, pdf, svg, bmp, cr2, gif, jpg, jpeg, pbm, png, ppm, tga, tiff, webp}으로 변환된다. 
디폴트는 \verb{pdf}이다.
GIF 이미지와 여러 페이지로 이루어진 PDF 파일은 낱개 이미지로 쪼개진다. 

\item[-T] 목표 포맷이 PNG인 경우에 흰색을 투명으로 바꾼다.

\item[-r] 화소 밀도를 조정하여 이미지의 인쇄 크기를 변경한다.

\item[-d] 해상도를 지정하라. 단위는 \term{ppc}\annotate{pixels per centimeter}이고, 디폴트는 100이다.

\item[-m] 아래 예의 경우에 가로 800 픽셀보다 큰 이미지들이 800 픽셀로 축소된다.
\begin{code}
C:\>iu -r -m 800 *.png
\end{code}

\item[-s] \verb{75}를 지정하면 원래 크기의 75 퍼센트로 바뀐다.

\item[-R] 모든 하위 폴더에 있는 이미지 파일들이 처리된다.

\item[-I] EPS를 PDF로 또는 그 반대로 변환하는 데에, \macro{epstopdf.exe}와 \macro{pdftops.exe} 대신, 잉크스케이프가 사용된다. 

\item[-c] \macro{pdfcrop.exe}가 PDF 이미지들에서 여백을 제거한다.
\end{macros}

\begin{note}
AI 파일은 PDF 또는 PDF를 거쳐 EPS로 변환된다.
반대로 변환하는 것은 가능하지 않다.
\end{note}

\section{fontinfo.py: 폰트 정보 보기}\label{sec:fontinfo}

\term{fontinfo.py}는 텍 라이브에 포함된 \term{fc-list.exe}와 \term{otfinfo.exe}를 이용하여 설치된 폰트들에 대한 정보를 보여준다.
사용 방법은 다음과 같다.

\begin{enumerate}

\item 텍 라이브를 포함하여 설치된 모든 폰트들의 목록이 \macro{fonts_list.txt}에 만들어진다.
출력 파일의 이름을 바꾸려면 \verb{-o} 옵션을 사용하라.

\begin{code}
C:\>fontinfo 
C:\>fontinfo -o myfonts.txt
\end{code}

\item 폰트 이름 또는 파일 이름을 지정하면 같은 이름의 PDF가 만들어진다. 
그 파일의 첫 페이지는, 어떤 언어들에 그 폰트를 사용할 수 있는지 보여주기 위한 목적으로, 언어별로 같은 의미의 짧은 문장들을 담고 있다.
나머지 페이지들은 그 폰트에 포함된 글자들을 유니코드 순으로 보여준다.
\fref{NotoSerifCJKKR.pdf}\를 보라.

\begin{code}
C:\>fontinfo "Noto Serif"
C:\>fontinfo NotoSerif-Regular.ttf
\end{code}

\item 폰트 정보를 보려면 \verb{-i} 옵션과 함께 폰트 이름 또는 파일 이름을 지정하라.

\begin{code}
C:\>fontinfo -i "Noto Serif"
C:\>fontinfo -i NotoSerif-Regular.ttf
\end{code}
\end{enumerate}

\image[float, frame, scale=0.75]{NotoSerifCJKKR.pdf}(Noto Serif CJK KR 폰트)

\section{tlconf.py: 텍 라이브와 관련된 설정}

텍 라이브를 처음 설치한 다음에 설정해야 할 것들이 몇 가지 있다.
\term{tlconf.py}는 텍 라이브를 위한 설정들을 수월하게 처리하고자 고안되었다.
\term{docenv.conf} 설정 파일에서 다음 옵션들이 적절하게 설정되어 있어야 한다.

\begin{codewrite}
[TeX Live]
texmflocal = C:\texlive\texmf-local\tex\latex\local
repository_main = https://cran.asia/tex/systems/texlive/tlnet/
repository_private = http://ftp.ktug.org/KTUG/texlive/tlnet

TEXEDIT = code.exe -r -g %%s:%%d
TEXMFHOME = C:\home\texmf

[LOCAL.CONF]
path = C:\texlive\2021\texmf-var\fonts\conf\local.conf
content = <dir>C:/Users/.../AppData/Local/Microsoft/Windows/Fonts</dir>

[SumatraPDF]
path = C:\Program Files\SumatraPDF\SumatraPDF.exe
inverse-search = code.exe -r -g %%f:%%l
\end{codewrite}
\coderead[language=ini] 

\begin{code}
C:\>tlconf -b
\end{code}

\begin{macros}
\item[-L] \term{local.conf} 파일이 만들어지고 사용자의 폰트 폴더 경로가 그 파일에 추가된다.
텍 라이브 업데이트가 간혹 이 파일을 삭제할 수 있다.
\item[-H] \term{TEXMFHOME} 환경 변수가 만들어지고 설정 파일에 지정되어 있는 경로로 설정된다.
\item[-e] \term{TEXTEDIT} 환경 변수가 만들어진다.
\item[-p] \term{Sumatra PDF}의 경로와 \term{inverse search} 옵션이 설정된다. 
\item[-r] \term{tlmgr.exe}가 사용할 텍 라이브 저장소가 지정된다.
\item[-u] 텍 라이브가 업데이트된다.
\item[-c] xelatex을 위해 \term{fc-cache}가 폰트들을 캐시한다.
폰트 캐시가 반드시 필요한 것이 아니다.
새로 설치되어, 아직 캐시되지 않은 폰트를 사용하는 경우에 텍 파일이 컴파일되면서 그 폰트의 캐시가 만들어진다. 
드물지만 폰트 캐시가 손상되는 경우가 있다. 
그 경우에 모든 폰트의 캐시를 새로 만드는 것이 해법이다.
\item[-l] lualatex을 위해 \term{luaotfload-tool}이 폰트 데이터베이스를 갱신한다.
\item[-f] 모든 포맷 파일들이 새로 생성된다. 이것이 64비트 lualatex이 정상적으로 작동하지 않을 때 해결책이 될 수 있다.
\item[-b] 이것은 위의 모든 옵션을 선택한 것과 동일하다.
\item[-q] 사용자 확인을 묻지 않고 작업들이 진행된다.
\end{macros}

HzGuide 클래스와 함께 제공되는 \term{tlconf.ps1} 및 \term{tlconf.cmd}는 tlconf.py의 간단 버전이다.

\begin{codewrite}
C:\>tlconf.ps1 -texmfhome 
C:\>tlconf.ps1 -texedit
C:\>tlconf.ps1 -sumatrapdf
C:\>tlconf.ps1 -conf
C:\>tlconf.ps1 -batch
\end{codewrite}
\coderead

아무 인수도 주지 않으면 각 옵션에 대한 도움말이 나온다.

\begin{macros}
\item[-t, -texmfhome] \macro{TEXMFHOME} 환경 변수를 설정한다.
\item[-e, -texedit] \macro{TEXEDIT} 환경 변수를 설정한다.
\item[-s, -sumatrapdf] \term{Sumatra PDF}에 \term{인버스 서치}\annotate{inverse search} 옵션을 설정한다.
\item[-c, -conf] 사용자 폰트 폴더의 경로를 포함하는 \macro{local.conf}를 생성한다.
\item[-b, -batch] 이것은 위의 모든 옵션들을 지시한 것과 동일하다.
\end{macros}

% \coderead[file=tlconf.ps1, language=powershell, fontsize=\footnotesize]
% \coderead[file=tlconf.cmd, language=batch, fontsize=\footnotesize]

\section{op.py: 파일 열기}

탐색기에서 어떤 파일을 열기 위해 특정 프로그램과 연결하는 일은 다소 성가시다.
\term{op.py}는 docenv.conf에 지정된 프로그램들을 이용하여 파일들을 연다.

\begin{codewrite}
[SumatraPDF]
path = C:\Program Files\SumatraPDF\SumatraPDF.exe

[Text Editor]
path = code.exe
associations = .aux, .bib, .bst, .cls, .cnf, .conf, .cmd, .css, .csv, .dita, .ditamap, .db, .gif, .gv, .ini, .ind, .idx, .ist, .ipynb, ,.jpg, .jpeg, .list, .lof, .log, .lot, .md, .png, .ps1, .py, .rst, .sty, .tex, .tmp, .toc, .tsv, .txt, .xml, .xsl, .xslt

[Adobe Reader]
path = C:\Program Files (x86)\Adobe\Acrobat Reader DC\Reader\AcroRd32.exe

[Web Browser]
path = C:\Program Files\Mozilla Firefox\firefox.exe
\end{codewrite}
\coderead[language=ini] 

아래 예처럼 대상들을 크게 다섯 가지로 나눌 수 있다.

\begin{code}
C:\>op foo.tex
C:\>op foo.pdf
C:\>op -s memoir.cls 
C:\>op -w ktug.org
C:\>op foo.docx
\end{code}

파일 이름의 일부만 지정할 수 있다. 그러면 지정된 문자들이 이름에 포함된 모든 파일들이 나열되고, 그 중 하나 또는 전부를 선택할 수 있다.
명시되지 않은 확장자의 파일을 여는 것은 윈도즈가 결정한다.

\begin{macros}
\item[-a] 달리 사용할 응용 프로그램을 지정하라.

\begin{code}
C:\>op -a notepad foo.txt
\end{code}

\item[-o] \verb{-a} 옵션과 함께, 응용 프로그램에 전달할 옵션들을 지정하라.
\item[-r] 하위 폴더들을 포함하여 지정된 것과 같은 이름의 파일들이 열린다.

\begin{code}
C:\>op -r foo*.xml
\end{code}

\item[-A] PDF 파일을 여는 데에 아크로뱃 리더가 사용된다.
\item[-s] 텍 라이브에 포함된 파일을 열려면 이 옵션을 사용하라. 파일이 열릴 때 그 파일의 폴더 경로가 클립보드에 복사된다.
\item[-f] 이미지 파일 같은 바이너리 파일을 포함하여, 지정되지 않은 형식의 파일을 텍스트 에디터로 열려면 이 옵션을 사용하라.
\item[-d] 주어진 파일을 그 파일의 확장자와 연결된 응용 프로그램이 연다. 이를테면 PNG 파일이 VS Code가 아닌 "사진" 프로그램에 열린다.
\item[-w] 웹 브라우저가 주어진 주소로 접속한다.
\item[-W] HTML 파일이나 XML 파일을 웹 브라우저로 열려면 이 옵션을 사용하라.
\end{macros}

\section{wordig.py: 문자열을 찾아 바꾸기}

\term{wordig.py}는, 단어 수 세기를 비롯하여, 레이텍 사용자가 텍스트 파일을 갖고 할 법한 많은 일들을 처리한다.

\begin{macros}
\item[-a] 찾을 문자열 또는 정규 표현식을 지정하라.
\item[-A] 한 줄에 하나씩, 찾을 문자열 또는 정규 표현식이 포함된 파일을 지정하라. 아래 변환 명세 파일의 규칙에 여기에 적용되지 않는다.
\item[-s] 바꿔치기할 문자열을 지정하라.
\item[-P] 변환 명세 파일을 지정하라. \term{TSV} 또는 \term{CSV} 형식으로 작성된 파일에서 변환 명세 파일에서 찾기--바꾸기를 나타내는 한 쌍의 정규 표현식과 치환 문자열이 각 줄을 차지해야 한다.
\term{변환 명세 파일}에 다음과 같은 규칙이 적용된다.
	\begin{itemize}
	\item 두 개의 칼럼이 존재한다.
	\item 탭 또는 콤마로 칼럼이 구분된다.
	\item 첫 칼럼에 찾을 문자열이, 둘째 칼럼에 대체 문자열이 있다. 둘째 칼럼이 없거나 비어 있으면 찾은 문자열이 삭제된다.
	\item 빈 줄이 무시된다.
	\item \verb{~~~} 또는 \verb{```}로 시작하는 줄이 주석으로 간주된다.
	\item 정규 표현식에서 도트(.)는 개행 문자를 제외한 모든 문자에 대응된다. 주석 줄 끝에 \macro{DOTALL}이 있으면, 그 다음 주석 줄을 만날 때까지 그 아래에 있는 모든 정규 표현식들의 처리에서 도트가 개행 문자를 포함하는 모든 문자에 대응된다. 
	\end{itemize}
\item[-c] 찾기에서 대소문자를 구분한다. 바꾸기를 위한 찾기에서는 항상 대소문자를 구분한다.
\item[-C] 두 파일을 비교하여 일치하지 않는 줄들을 찾아낸다.

\begin{code}
C:\>wordig -C C:\project\A\foo.xml C:\project\B\goo.xml
C:\>wordig -C -R -O comp.txt C:\project\A\*.xml C:\project\B\
\end{code}


\item[-L] 이 옵션이 주어지면 도트가 개행 문자를 포함하는 모든 문자에 대응된다. 
\item[-e] 주어진 정규 표현식에 일치하는 문자열들을 추출하여 파일별로 저장한다. \verb{-a} 또는 \verb{-A} 옵션이 설정되어야 한다.
\item[-g] 문자열들을 추출하고 한 파일에 합쳐 저장한다. 다음 예가 모든 텍 파일에서 단어들을 추출한다. 달리 지정하지 않으면 출력 파일의 이름이 \verb{gathered_strings.txt}이다.

\begin{code}
C:\>wordig -a="\w+" -g *.tex
\end{code}

\item[-o] 출력 파일을 위한 이름을 지정하라. 디폴트는 기능에 따라 달라진다. 대상 파일의 이름이 \verb{foo.tex}일 때, 주어진 이름에 따라 출력 파일의 이름이 다음과 같이 결정된다. 

\begin{codewrite}
-o "_replaced" → foo_replaced.tex
-o "replaced_" → replaced_foo.tex
-o ".txt" → foo.txt
-o "goo" → goo.tex
-o "goo.txt" → goo.txt
-o "sub\" → sub\foo.tex
-o "sub\_replaced" → sub\foo_replaced.tex
\end{codewrite}
\coderead|

\term{폴더 구분자}\annotate{directory separator}로 백슬래시(\verb{\}) 대신 슬래시(\verb{/})를 사용할 수 있다.

\item[-v] 이 옵션이 주어졌을 때 출력 파일의 이름이 지정되지 않으면 ---\verb{-o} 옵션이 주어지지 않으면--- 같은 이름의 파일이 존재할 때 출력을 위한 파일 이름에 번호가 추가된다. 
다른 기능이 수행될 때에도 마찬가지이다. 문자열 치환이 그 결과를 원본 파일에 덮어 쓴다. 
\item[-r] 모든 하위 폴더에 있는 파일들이 처리된다.
\item[-p] 각 PDF 파일들의 페이지 수와 합계를 구하려면 이 옵션을 사용하라.
\item[-U] 주어진 문자들의 유니코드 정보가 표시된다.

\begin{code}
C:\>wordig -U 가나
\end{code}

\item[-u] 주어진 문자들의 유니코드 정보와 함께 UTF-8 비트들이 표시된다.
\item[-X] 주어진 16 진수 값의 유니코드 문자가 표시된다.

\begin{code}
C:\>wordig -X AC00 B098
\end{code}

\item[-D] 주어진 10 진수 값의 유니코드 문자가 표시된다.
\item[-E] 대상 파일들이 지정된 인코딩에서 UTF-8로 변환된다. 달리 지정하지 않으면 \verb{UTF-8} 폴더가 만들어지고 그 안에 출력 파일들이 저장된다.

\begin{code}
C:\>wordig.py -E cp949 *.txt
\end{code}

\item[-x] .tsv 파일을 .xlsx로 변환한다.
\item[-t] .xlsx 파일을 .tsv로 변환한다. \ptref{chp:script_application}\을 보라.
\end{macros}

아무 옵션도 지시하지 않으면 주어진 파일들의 글자 수와 단어 수가 표시된다.

\begin{code}
C:\>wordig *.tex
C:\>wordig *.pdf
\end{code}

문자열을 찾거나 바꾸는 방법은 다음과 같다.

\begin{code}
C:\>wordig -a "itemize" *.tex
C:\>wordig -a "network" *.pdf
C:\>wordig -a "itemize" -s "enumerate" *.tex
C:\>wordig -P foo.tsv *.tex
\end{code}

다음 예가 자동조사 매크로를 조사로 또는 반대로 바꾼다.

\begin{code}
C:\>wordig -a="\\([은는이가을를와과으로])" -s="\1" *.tex
C:\>wordig -a="(ref.+?\})([은는이가을를와과으로])" -s="\1\\\2" *.tex
\end{code}

텍 매크로들을 제거하기 위한 목적으로 \term{detex.tsv}가 작성되었다.

\begin{codewrite}
(?<!\\)%.*?$
\\include.+$
\\begin\{.+\}.+$
\\end\{.+\}
\\[a-zA-Z*]{1,25}
\end{codewrite}
\coderead|

\begin{code}
C:\>wordig -P detex.tsv -o "_cleaned" *.tex 
\end{code}

텍 매크로들을 추출하려면 다음과 같이 정규 표현식 파일을 작성한다.

\begin{codewrite}
\\[^a-zA-Z]
\\[a-zA-Z*^|+]+
\\begin(\{.+?\}[*^|+]*)
\end{codewrite}
\coderead|

\begin{code}
C:\>wordig -A extractex.txt -g *.tex 
\end{code}

\section{unit.py: 단위 변환하기}

\term{unit.py}는 다음과 같은 여러 비표준 단위의 값들을 미터법 단위로 환산한다.\annotate*{미국이 미터법을 채용하지 않은 거의 유일한 나라이다.}

\begin{codewrite}
acre, degree, fahrenheit, foot, gallon, hp, inch, km/h, knot, mile, Newton, pascal, pint, point, pound, psi, pyeong, yard
\end{codewrite}
\coderead|

다른 단위와 구분될 수 있을 만큼 사용할 단위의 처음 글자들을 숫자와 함께 지정하라.
아래 예에서 첫 줄의 \verb{p}는 \verb{pascal}로, 둘째 줄의 \verb{po}는 \verb{point}로 간주된다.

\begin{code}
C:\>unit 10 p 
C:\>unit 20 po
C:\>unit 30 pou
C:\>unit 30 py
\end{code}

천 자리 구분자로 쉼표를 사용할 수 있는데, 파워셸에서는 따옴표로 값을 감싸거나 이스케이프 문자인 억음 부호와 함께 쉼표를 써야 한다.

\begin{code}
C:\>unit 2`,000`,000 mi
C:\>unit "2,000,000" mi
\end{code}

unit.py는 또한 RGB 색상을 CMYK로 또는 그 반대로 바꿀 수 있다.
RGB는 HTML 색상 값이나 16 비트 10 진수로, CMYK는 백분율 값으로 지정해야 한다.
띄어쓰기를 하려면 값을 따옴표로 감싸야 한다.

\begin{code}
C:\>unit -c 0000FF
C:\>unit -c 240,120,99
C:\>unit -c '240, 120, 99'
C:\>unit -c "0, 0.1, 0.33, 0.01"
\end{code}

10 진수의 16 진수를 또는 16 진수의 10 진수를 얻으려면 -x 옵션을 사용한다.

\begin{code}
C:\>unit -x 44032
C:\>unit -x AC00
\end{code}

\section{fu.py: 파일들 백업하기}

fu.py는 파일들을 관리하기 위한 목적으로 고안되었다.

\begin{macros}
\item[-d, --destination] 목적 폴더를 지정하라. 디폴트가 기능에 따라 달라진다.
\item[-r, --rename-files] 파일 이름을 변경한다.
\item[-a, --affix] 파일 이름을 변경하기 위해 찾을 문자열 또는 정규 표현식을 지정하라.
\item[-s, --substitute] 파일 이름을 변경하기 위해 바꿔치기할 문자열을 지정하라.
\item[-g, --gather-files] 하위 폴더에 있는 것들을 포함하여, 파일들을 지정된 폴더로 복사한다.
\item[-t, --total-size] 하위 폴더에 있는 것들을 포함하여, 모든 파일들의 전체 크기를 구한다.
\item[-f, --flag] 각 기능에 사용할 방법을 지정하라.
\end{macros}

\subsection{파일들 백업하기}

목적 폴더를 지정하지 않으면, \verb{_bak} 폴더가 만들어지고, 그 아래에 파일들이 복사된다.
\verb{-a}나 \verb{--affix} 옵션을 사용하여 파일들의 이름에 붙일 문자열을 지정할 수 있다.
달리 지정하지 않으면 오늘 날짜가 접사로 사용된다.

\begin{code}
C:\>fu -d=c:\repository -a=adhoc *.tex *.pdf
\end{code}

같은 이름의 파일이 존재한다면 번호가 추가된다.

\begin{code}
foo_2020-10-10.tex
foo_2020-10-10_1.tex
...
foo_2020-10-10_2.pdf
\end{code}

\verb{-f} 또는 \verb{--flag} 옵션을 사용하여 백업 방식을 지정할 수 있다.

\begin{macros}
\item[0, file-today] 파일 이름에 접사를 붙인다. 이것이 디폴트이다.

\begin{code}
C:\>fu *.tex
_bak\foo_2020-10-10.tex
\end{code}

\item[1, directory-today] 접사와 같은 이름으로 하위 폴더를 만든다.

\begin{code}
C:\>fu -f=1 *.tex
_bak\2020-10-10\foo.tex
_bak\2020-10-10_1\foo.tex
\end{code}

\item[2, directory-file-today] 접사와 같은 이름으로 하위 폴더를 만르고, 파일 이름에 접사를 붙인다.

\begin{code}
C:\>fu --flag=directory-file-today *.tex
_bak\2020-10-10\foo_2020-10-10.tex
_bak\2020-10-10_1\foo_2020-10-10.tex
\end{code}
\end{macros}

\subsection{파일 이름 바꾸기}

다음과 같이 파일 이름 변경을 위한 다양한 플래그 옵션들이 마련되어 있다.

\begin{macros}
\item[0, append-letters] 접사를 파일 이름 끝에 붙인다. 달리 지정하지 않으면 오늘 날짜가 접사로 사용된다.

\begin{code}
C:\>fu -r *.pdf
foo.pdf -> foo_2020-10-10.pdf
\end{code}

\item[1, prepend-letters] 접사를 파일 이름 머리에 붙인다. 

\begin{code} 
C:\>fu -r -f=1 *.pdf
foo.pdf -> 2020-10-10_foo.pdf
\end{code}

\item[2, remove-letters] 파일 이름에서 접사를 제거한다. 

\begin{code} 
C:\>fu -r --flag=remove-letters --affix="_2020-10-10" *.pdf
foo_2020-10-10.pdf -> foo.pdf
\end{code}

\item[3, replace-letters] 파일 이름으로 접사를 지정된 문자열로 바꾼다.

\begin{code} 
C:\>fu -r --flag=replace-letters --affix="&" --suffix="#" *.pdf
foo&goo.pdf -> foo#goo.pdf
\end{code}

\item[4, remove-spaces] 파일 이름에서 공백을 제거한다.

\begin{code} 
C:\>fu -r -f=remove-spaces *.pdf
Korean User Guide.pdf -> KoreanUserGuide.pdf
\end{code}

\item[5, uppercase] 파일 이름에서 모든 로마자를 대문자로 바꾼다.

\begin{code} 
C:\>fu -r -f=uppercase *.pdf
user_guide.pdf -> USER_GUIDE.PDF
\end{code}

\item[6, lowercase] 파일 이름에서 모든 로마자를 소문자로 바꾼다.

\begin{code} 
C:\>fu -r -f=lowercase *.pdf
USER_GUIDE.PDF -> user_guide.pdf 
\end{code}

\item[7, ext-lowercase] 파일 확장자만 소문자로 바꾼다.

\begin{code} 
C:\>fu -r -f=ext-lowercase *.png *.jpg
Image_1.PNG -> Image_1.png
Image_2.JPG -> Image_2.jpg
\end{code}

\item[8, date-created] 사진 파일의 경우에 파일 이름을 이미지가 생성된 (촬영된) 날짜로 바꾼다.
    메타데이터가 없는 사진 파일들과 다른 포맷의 파일들에 대해서는 마지막 수정 날짜가 사용된다.

\begin{code} 
C:\>fu -r -f=date-created *.jpeg
IMG_1215.JPEG -> 2022-06-25_01.JPEG
IMG_1216.JPEG -> 2022-06-25_02.JPEG
\end{code}
\end{macros}

\subsection{파일들 모으기}

그다지 생길 법한 일이 아니지만, 하위 폴더에 있는 것들을 포함하여, 특정 형식의 파일들을 한 군데로 모아야 할 때가 있다.

\begin{code}
C:\>fu -g d=c:\repository c:\projects\*.pdf
\end{code}

달리 지정하지 않으면 현재 폴더로 파일들이 복사된다.

\begin{macros}
\item[0, overwrite] 같은 이름의 파일이 있으면 덮어 쓴다.
\item[1, append-number] 같은 이름의 파일이 있으면 파일 이름에 번호를 붙인다.

\begin{code}
C:\>fu -g -f=append-number *.pdf

.\foo.pdf
.\foo_1.pdf
\end{code}
\end{macros}

\subsection{전체 파일 크기 구하기}

파일들이 복사되는 도중에 뜨는 공간이 부족하다는 메시지가 결코 유쾌하지 않다.
폴더를 복사하기 전에 파일들의 전체 크기를 알아내는 것이 현명하다.

\begin{code}
C:\>fu -t c:\foo d:\goo

c:\foo\ 12.34 MB
d:\goo\ 5.6 GB
\end{code}

달리 지정하지 않으면, 현재 폴더 아래에 있는 모든 파일들의 크기를 합산하다.

\section{trapper.py: 스크린샷 저장하기}\label{sec:trapper}

\winkey\ + \button{Shift} + \button{S} 키를 이용하여 스크린샷을 캡처하고 클립보드에 복사할 수 있다.
그러나 캡처한 이미지를 파일로 저장하려면 캡처하기 전에 "\term{캡처 및 스케치}"를 열거나 캡처한 뒤에 "\term{그림판}" 따위를 열어야 한다.
그 번거로움을 피하기 위해 \term{trapper.py}가 작성되었다.
\term{trapper.py}는 클립보드에 있는 이미지를 파일로 저장한다. 

\begin{code}
C:\>trapper [foo]
C:\>trapper -o foo
C:\>trapper -c foo.png
\end{code}

파일 이름을 지정하지 않으면, \verb{2022-10-25_2}와 같이 ---같은 이름의 파일이 존재하면 번호를 추가하여--- 오늘 날짜가 파일 이름이 된다. 
PNG 포맷으로만 저장된다. 클립보드에 있는 것이 이미지가 아니라 텍스트라면 그 텍스트가 출력된다.
\verb{-o} 옵션을 주면 기존 파일을 덮어 쓴다.
\verb{-c} 옵션을 주면, 반대로, 지정된 이미지 파일이 클립보드에 복사된다.

\section{레이텍 템플릿 모음}\label{sec:mytex}

여러 레이텍 예들이 \term{latex.db}라는 데이터베이스 파일에 저장되어 있다.
\term{mytex.py}가 \mbox{latex.db}에서 지정된 템플릿을 꺼내어 저장하고 컴파일한다.
다음은 latex.db의 일부이다.

\begin{codewrite}
[japanese]
description = This template shows how to typeset Japanese.
compiler = -L
tex_output = japanese.tex
tex = ```\documentclass{ltjbook}
    ```\begin{document}
    ```\tableofcontents
    ```\chapter{本機器}
    ```\section{運用寿命}
    ```ハネウェルAnalyticsは、本機器の運用寿命の期間、通常の使用及びサービスの下で、製品に材料および製造上の欠陥がないことを保証します。
    ```\end{document}
\end{codewrite}
\coderead[language=ini] 

대괄호로 감싼 말이, ini 형식에서 섹션을 가리키는데, 템플릿을 나타낸다.

\begin{code}
C:\>mytex japanese
\end{code}

\term{japanese} 템플릿이 ltjbook 클래스를 이용하여 일본어를 조판하는 방법을 보여준다.
\verb{compiler} 옵션에 설정된 컴파일 조건들이 ltx.py에 전달된다.

\image[float, frame, scale=0.75]{japanese.pdf}(japanese 템플릿)

\begin{macros}
\item[-o] 출력 파일의 이름을 지정하라.
\item[-s] \macro{placeholders} 옵션이 지정된 템플릿에서 \verb{\1}처럼 백슬래시로 시작하는 숫자가 이 옵션과 함께 주어진 문자열로 대체된다. 

\begin{code}
C:\>mytex -s "20, 10" lotto
\end{code}

\item[-n] \macro{compiler}에 설정된 옵션들이 ltx.py에 전달되는데, \verb{-n}이 주어지면 ltx.py가 호출되지 않는다.
\item[-f] \macro{compiler} 옵션에 아무 것도 설정되어 있지 않아도 템플릿 파일이 ltx.py에 의해 컴파일된다.
\item[-d] PDF 파일을 제외하고 템플릿 파일을 비롯하여 부수적으로 생성된 모든 파일들이 컴파일 뒤에 삭제된다.
\item[-l] 템플릿들이 나열된다.
\item[-L] 각 템플릿의 설명과 함께 템플릿들이 나열된다.
\item[-D] 지정된 템플릿에 대한 상세한 설명이 표시된다.
\item[-i] 지정된 텍 파일이 데이터베이스에 삽입된다. 텍 파일의 이름이 템플릿 이름이 된다.

\begin{code}
C:\>mytex -i foo.tex style_output=foo.sty image_output=foo.jpg
\end{code}

텍 파일에 포함된 다음과 같은 주석들이 템플릿 옵션으로 해석된다.

\begin{codewrite}
% description =  An example of how to ...
% compiler = -c
% placeholders = 1
% defaults = foo
\documentclass{oblivoir}
...
\end{codewrite}
\coderead[language=latex]

다른 파일들을 함께 저장하려면 위 예와 같이 접미어 \verb{_output}이 포함된 이름과 함께 파일을 지정하라.
그러면 데이터베이스에 다음과 같이 추가될 것이다.

\begin{codewrite}
[foo]
tex_output = foo.tex
tex = \documentclass{...}
style_output = foo.sty
style = \ProvidesPackage{...}
\end{codewrite}
\coderead[language=ini]

\item[-u] 추출된 텍 파일을 수정하고 그것으로 데이터베이스를 업데이하려면 이 옵션을 사용하라.

\begin{code}
C:\>mytex -u multilingual
\end{code}

\item[-r] 데이터베이스에서 지정한 템플릿이 제거된다.
\item[-b] 모든 템플릿들이 추출된다.
\item[-R] 이것은 \macro{album} 템플릿을 위한 옵션이다. 하위 폴더들에 포함된 이미지 파일들을 이미지 목록에 추가한다.
\end{macros}

mytex.py가 이해하지 못하는 옵션들은 ltx.py에 전달된다

\section{이따금 요긴한 스크립트들}\label{sec:myscript}

\term{scripts.db}라는 데이터베이스 파일에 가끔 유용할 수 있는 스크립트들이 들어 있다. 
대부분이 파이선으로 작성되었고, 일부는 파워셸로 작성되었다.
\term{myscript.py}를 이용하여 특정 스크립트를 꺼내어 실행할 수 있다.
다음 예에서 scripts.db로부터 gencode.py가 추출되고, gencode.py가 KTUG 웹 주소를 가리키는 QR 코드를 만든다.

\begin{code}
C:\>myscript gencode
C:\>gencode -q www.ktug.org
\end{code}

\begin{macros}
\item[-I] 새로 작성한 스크립트 파일을 scripts.db에 삽입한다.

\begin{code}
C:\>myscript -I foo.py data_output=foo.tsv
\end{code}

다른 파일들을 함께 저장하려면 위 예와 같이 `\verb{_output}'이 포함된 이름과 함께 파일을 지정하라.
그러면 데이터베이스에 다음과 같이 추가될 것이다.

\begin{codewrite}
[foo]
code_output = foo.py
code = import ... 
data_output = foo.tsv
data = ...
\end{codewrite}
\coderead[language=ini]

\item[-U] 추출된 스크립트 파일을 수정하고 그것으로 scripts.db를 업데이트하려면 이 옵션을 사용하라.

\begin{code}
C:\>myscript -U foo
\end{code}

\item[-R] 지정된 스크립트가 데이터베이스에서 제거된다.
\item[-B] 모든 스크립트들이 추출된다.
\item[-C] 실행 뒤에 추출된 스크립트가 삭제된다.
\end{macros}

\chapter{응용}\label{chp:script_application}

소프트웨어 제품에 대한 매뉴얼을 작성할 때 짐스러운 것이 UI 문자열이다.
대개 개발자들이 모든 UI 문자열을 담은 엑셀 파일을 제공한다.
그 파일은 매뉴얼 번역에 필수적이다.
예를 들어 "지금 설치" 버튼이 스페인어로 옮길 때 "Actualizar ahora"로 바뀌어야 한다.

\image{UI_xlsx.png}

실제 화면을 확인할 수 없는 상황에서도 UI 문자열 파일이 유용하다.
하지만 엑셀 파일이 너무 많은 언어들을 포함하고 있다면 이용하기에 불편하다.
wordig.py와 \macro{\ReadTSV} 명령을 이용하여 그 불편을 크게 줄일 수 있다.

\begin{enumerate}
\item wordig.py를 이용하여 엑셀 파일에서 특정 칼럼들을 추출하여 UI.tsv 파일로 저장한다.
\item \verb{\ReadTSV{UI.tsv}}를 포함하는 UI.tex을 작성하여 PDF를 만든다. \ptref{sec:readtsv}\를 보라.
\item wordig.py를 이용하여 만들어진 PDF 파일에서 문자열을 검색한다.
\end{enumerate}

\begin{code}
C:\>wordig -t -a="1,3-5" -o=foo.tsv foo.xlsx
\end{code}

\verb{-a} 옵션을 사용하여 추출할 칼럼들을 지정할 수 있다.
아무 것도 지정하지 않으면 모든 칼럼들이 추출된다.

\begin{codewrite}
\documentclass[9pt]{hzguide}
\LayoutSetup{}
\HeadingSetup{article, parskip=5pt}
\setmainfont{HANDotum.ttf}
\RenewDocumentCommand\texttsv{m}{{\footnotesize #1}}
\NewDocumentCommand\colbox{m}{\parbox[t]{0.42\textwidth}{\raggedright\small #1}}
\TSVsetup{
	tabs={0pt, 0.12\textwidth, 0.56\textwidth},
	renderers={\texttsv, \colbox, \colbox},
	interline={\dotfill\linebreak},
	verbatim=false
}
\begin{document}
\ReadTSV{\1}
\end{document}
\end{codewrite}
\coderead[language=latex]

이 레이텍 문서를 latex.db에 저장하면 언어별로 다수의 PDF 파일들을 보다 쉽게 만들 수 있다.
\ptref{sec:mytex}\을 보라.

\begin{code}
C:\>mytex -i UI.tex
\end{code}

그리고 다음과 같은 명령을 포함하는 xlsx2pdf.cmd를 만든다.

\begin{codewrite}
wordig.py -t -a %3 -o %~n2.tsv %1
mytex.py -f -o %~n2.tex -s %~n2.tsv UI
\end{codewrite}
\coderead[language=batch]

\begin{code}
C:\>xlsx2pdf.cmd UI.xlsx UI_kor_eng "1-3"
C:\>xlsx2pdf.cmd UI.xlsx UI_eng_fre "1,3,4"
C:\>xlsx2pdf.cmd UI.xlsx UI_eng_ger "1,3,5"
\end{code}

\begin{note}
파워셸에서는 cmd 명령에 인수를 전달하기 위해 따옴표에 억음 부호를 추가해야 한다. (\macro*{`"1-3`"})
\end{note}

그러면 \verb{UI_kor_eng.pdf}, \verb{UI_eng_fre.pdf}, \verb{UI_eng_ger.pdf}가 만들어질 것이다.

\image{xlsx2pdf.png}

PDF 파일을 열지 않고 wordig.py를 이용하여 문자열을 검색할 수 있다.

\begin{code}
C:\>wordig -a "I have a dream" UI.pdf
\end{code}

\image{search_pdf.png}

